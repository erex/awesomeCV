\cvsection{Observing Programmes}

\begin{cvpubs}
  \cvpub
    {
      \begin{cvlist}
        \item {\textbf{Co-I: e--MERLIN (PI: J. Greaves, CY5214, 330 hours)}, Planet-Earth Building Blocks - a Legacy e--MERLIN Survey.}
  \item {\textbf{Co-I: e--MERLIN (PI: J. Greaves, CY4211, 12 hours)}, What planets for DG Tau?: Observations of DG~Tau at 21--24\,GHz to investigate dust concentration in the circumstellar disk.}
  \item {\textbf{PI: VLA Cycle 2016A (16A-051, 6.5 hours)}, Confirming Cosmic Ray Production in a Protostellar Jet: Observations of DG Tau at C and X-band, C-config to measure the bow shock proper motion.}
  \item {\textbf{PI: LOFAR Cycle 5 (LC5\_004, 8 hours)}, VLBI Investigations of a Protostellar Jet with LOFAR: Low-frequency, high-resolution observations of T~Tau.}
  \item {\textbf{PI: VLA Cycle 2015A (15A-143, 2 hours)}, Cosmic Rays Generated in the Jet of a Young Sun-like Star?: Observations of DG Tau at S-band, A-config to confirm synchrotron nature of bow shock.}
  \item {\textbf{PI: VLA Cycle 2014A (14A-439, 14 hours)}, Polarisation Measurements of Protostellar Jets: Observations of 3 YSOs at L and S-band, A-config to detect linearly polarised emission.}
  \item {\textbf{PI: VLA Cycle 2014A (14A-457, 6 hours)}, Radio Continuum Observations of FU Orionis Stars: Observations of 4 FUors at X and Ku-band, A-config to detect individual ejection episodes.}
  \item {\textbf{Co-I: LOFAR Cycle 1 (PI: J. Eisl{\"o}ffel, LC1\_001, 17 hours)}, Low Frequency Observations of Jets from Young Stars in Taurus: To follow up the low frequency GMRT observations at 150\,MHz, confirm the emission mechanism at low frequency, and study outflow structure.}
  \item {\textbf{PI: GMRT Cycle 25 (25\_072, 22 hours)}, Low Frequency Radio Emission from the Youngest Low Mass Protostars: To extend the GMRT pathfinder program to Class~0 objects at 325 and 610\,MHz.}
  \item {\textbf{PI: GMRT Cycle 25 (25\_066, 58 hours)}, Blind Survey of the NGC~1333 Star Forming Region at Low Frequencies: To perform a radio census of Class~0--III YSOs at 610\,MHz.}
      \end{cvlist}
    }
\end{cvpubs}