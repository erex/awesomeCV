\cvsection{Experience}
\begin{cventries}
  \cventry
    {Research Associate}
    {Jodrell Bank Centre for Astrophysics (JBCA), University of Manchester}
    {Manchester, UK}
    {June 2017 - present}
    {
      \begin{cvitems}
        \item {Developing standardised software for calibrating dispersive delay corrections in long baseline interferometry, for low frequency radio telescopes such as LOFAR, as part of the Horizon 2020 RadioNET ``Radio Interferometry Next Generation Software'' (RINGS) project.}
        \item {Conducting research to exploit the polarisation capabilities of the e--MERLIN telescope.}
        \item {Interferometry Centre for Excellence (ICE) Open Science Champion to promote, advocate and organise events relating to open science in astronomy.}
      \end{cvitems}
    }
\cventry
    {Postdoctoral Research Fellow}
    {Dublin Institute for Advanced Studies (DIAS)}
    {Dublin, Ireland}
    {October 2014 - October 2016}
    {
      \begin{cvitems}
        \item {Member of a team to develop novel processing and analytical techniques for terabytes of data from the International LOFAR Telescope to detect very faint radio sources with this next-generation instrument.}
        \item {Collaborated with Th{\"u}ringer Landessternwarte in the use of LOFAR data to help understand the generation of outflows from young stars: achieved the first detection of a young stellar object (YSO) at 2\,m (150\,MHz).}
        \item {Member of the Communications Working Group to re-define the public communication strategy of DIAS through restructure of the website and social media developments.}
      \end{cvitems}
    }
  \cventry
    {Ph.D. Student; Supervisors: Prof. Tom Ray (DIAS), Dr. Anna Scaife (JBCA)}
    {}
    {}
    {November 2010 - September 2014}
    {
      \begin{cvitems}
        \item {Led three projects performing systematic modelling of multi-wavelength, multi-scale datasets of protostellar jets from AMI, e--MERLIN \& GMRT to disentangle competing radiation processes and investigate the jet launching mechanism.}
        \item {Involved in the commissioning of e--MERLIN, which included the reduction and analysis of legacy data (Thermal Jets, PEBBLeS) intermediate to the original MERLIN and the fully upgraded e--MERLIN.}
        \item {Published the first investigations of YSOs at metre wavelengths and pioneered to characterise this very long wavelength emission through follow-up observing campaigns on the GMRT and LOFAR.}
      \end{cvitems} 
    }
\cventry
    {Summer Research Fellow; Supervisor: Dr. Michael Guidry (UT)}
    {University of Tennessee (UT)}
    {Knoxville, TN, USA}
    {May 2009 - August 2009}
    {
      \begin{cvitems}
        \item {Studied the explosion mechanism of Type Ia Supernovae through computational simulations using the FLASH code on the supercomputing facilities at Oak Ridge National Laboratory.}
      \end{cvitems}
    }
\cventry
    {Undergraduate Student Researcher; Supervisor: Dr. Michael Guidry (UT)}
    {}
    {}
    {June 2008 - August 2008}
    {
      \begin{cvitems}
        \item {Studied the interactions between colliding galaxies using computational simulations developed by the UT Astrophysics Group.}
      \end{cvitems}
    }
\cventry
    {Undergraduate Student Research Program Intern; Supervisor: Dr. Raghvendra Sahai (JPL/Caltech)}
    {NASA Jet Propulsion Laboratory / California Institute of Technology (JPL/Caltech)}
    {Pasadena, CA, USA}
    {September 2008 - December 2008}
    {
      \begin{cvitems}
        \item {Developed and applied a procedure for the reduction and calibration of near-infrared echelle spectroscopic data for (a)  a sample of pre-planetary nebulae to look for the signatures of high-velocity outflows that shape the resulting planetary nebula, and (b) stellar interlopers: young stars with winds speeding through and interacting with dense interstellar clouds.}
      \end{cvitems}
    }
\end{cventries}
